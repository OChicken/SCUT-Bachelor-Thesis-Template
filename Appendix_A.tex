%----------------------------------------------------------------------------------------
%   我是附录
%----------------------------------------------------------------------------------------
\section{我是附录}
\label{sec:A}
像``附录A''及其页眉也是可以设置的.

为了展示奇偶页眉的效果, 将上面那句话复制若干次, 直到附录\ref{sec:A}写到第二页位置, 读者可以看看页眉的变化.

像``附录A''及其页眉也是可以设置的.

像``附录A''及其页眉也是可以设置的.

像``附录A''及其页眉也是可以设置的.

像``附录A''及其页眉也是可以设置的.

像``附录A''及其页眉也是可以设置的.

像``附录A''及其页眉也是可以设置的.

像``附录A''及其页眉也是可以设置的.

像``附录A''及其页眉也是可以设置的.

像``附录A''及其页眉也是可以设置的.

像``附录A''及其页眉也是可以设置的.

像``附录A''及其页眉也是可以设置的.

像``附录A''及其页眉也是可以设置的.

像``附录A''及其页眉也是可以设置的.

像``附录A''及其页眉也是可以设置的.

像``附录A''及其页眉也是可以设置的.

像``附录A''及其页眉也是可以设置的.

像``附录A''及其页眉也是可以设置的.

像``附录A''及其页眉也是可以设置的.

像``附录A''及其页眉也是可以设置的.

像``附录A''及其页眉也是可以设置的.

像``附录A''及其页眉也是可以设置的.

像``附录A''及其页眉也是可以设置的.

像``附录A''及其页眉也是可以设置的.

像``附录A''及其页眉也是可以设置的.

像``附录A''及其页眉也是可以设置的.

像``附录A''及其页眉也是可以设置的.

像``附录A''及其页眉也是可以设置的.

像``附录A''及其页眉也是可以设置的.

像``附录A''及其页眉也是可以设置的.

可以试试写个公式. 公式的编号自动为A.1
\begin{equation}\label{equ:A:1}
a^2+b^2=c^2
\end{equation}
并且
\begin{equation}\label{equ:A:2}
c^2=a^2+b^2
\end{equation}

\pagebreak[4]
