%----------------------------------------------------------------------------------------
%   Chinese Abstract
%----------------------------------------------------------------------------------------
\setcounter{page}{1}
\pagenumbering{Roman}
\begin{center}
\addcontentsline{toc}{section}{摘要} % Add to content
\zihao{-2}
\bfseries
 摘\quad 要
\end{center}
\thispagestyle{plain}
\begin{spacing}{1.5}
\zihao{-4}
摘要内容应概括地反映出本论文的主要内容, 主要说明本论文的研究目的、内容、方法、成果和结论. 要突出本论文的创造性成果或新见解, 不要与引言相混淆. 语言力求精练、准确,以300-500字为宜. 在摘要的下方另起一行, 注明本文的关键词 (3-5个). 关键词是供检索用的主题词条, 应采用能覆盖论文主要内容的通用技术词条 (参照相应的技术术语标准). 按词条的外延层次排列 (外延大的排在前面). 摘要与关键词应在同一页.
\ \\
\textbf{关键词: }多变量系统; 预测控制; 环境试验设备
\end{spacing}
\pagebreak[4]
\thispagestyle{plain}

%----------------------------------------------------------------------------------------
%   English Abstract
%----------------------------------------------------------------------------------------
\begin{center}
\addcontentsline{toc}{section}{Abstract} % Add to content
\zihao{-2}
Abstract
\end{center}
\begin{spacing}{1.5}
\zihao{-4}
英文摘要内容与中文摘要相同, 以 250-400个实词为宜. 摘要下方另起一行注明英文关键词 (Keywords3-5个).
\ \\
\textbf{Keywords: }Writer recognition; Convolutional Neural Network; Handwritten character recognition
\end{spacing}
\pagebreak[4]
