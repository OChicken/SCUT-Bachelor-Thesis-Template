%----------------------------------------------------------------------------------------
%   简单的使用例子
%----------------------------------------------------------------------------------------
\section{简单的使用例子}
\label{cha:example}
%----------------------------------------------------------------------------------------
%   图片的插入
%----------------------------------------------------------------------------------------
\subsection{图片的插入}
\label{sec:Images}
\begin{wrapfigure}{r}{0.5\linewidth}
\centering
\includegraphics[width=0.5\textwidth]{images/Chap2/confusion.pdf} % width=0.5\textwidth也可以换成width=5cm, height=5cm等关键字参数. width=0.5\textwidth 比较常用: 图片宽度等于你的text宽度的1/2
\caption{镶嵌在文中的图像}
\label{fig:confusion}
\end{wrapfigure}
论文主体是毕业论文的主要部分, 必须言之成理, 论据可靠, 严格遵循本学科国际通行的学术规范. 在写作上要注意结构合理、层次分明、重点突出, 章节标题、公式图表符号必须规范统一. 论文主体的内容根据不同学科有不同的特点, 一般应包括以下几个方面: (1)毕业论文 (设计) 总体方案或选题的论证; (2)毕业论文 (设计) 各部分的设计实现, 包括实验数据的获取、数据可行性及有效性的处理与分析、各部分的设计计算等; (3)对研究内容及成果的客观阐述, 包括理论依据、创新见解、创造性成果及其改进与实际应用价值等; (4)论文主体的所有数据必须真实可靠, 凡引用他人观点、方案、资料、数据等, 无论曾否发表, 无论是纸质或电子版, 均应详加注释. 自然科学论文应推理正确、结论清晰; 人文和社会学科的论文应把握论点正确、论证充分、论据可靠, 恰当运用系统分析和比较研究的方法进行模型或方案设计, 注重实证研究和案例分析, 根据分析结果提出建议和改进措施等.

可以插入单张图像:
\begin{figure}[h]
\centering
\includegraphics[width=0.5\textwidth]{images/Chap2/hole.pdf}
\caption{单张图像}
\label{fig:hole}
\end{figure}

甚至可以两张并排, 如图\ref{B-C_ratio}和图\ref{fig:gamma_MCMCanalysis}所示, 并且这句话已经实现了用$\backslash$ref\{key\}来引用内容. 引用的内容包括公式、表格、图片、代码段等等. 对于参考文献的引用并非用ref, 之后会提及.

对于多张图片的并排并统一标注caption可使用subfigure和makebox; 因为这个功能用得较少, 用户可自行Google.

\begin{figure}[th]
\parbox[t]{0.5\textwidth}{
\includegraphics[width=0.5\textwidth]{images/Chap2/B-C_ratio.pdf}
\caption{B/C ratio}
\label{B-C_ratio}
}
\parbox[t]{0.5\textwidth}{
\includegraphics[width=0.5\textwidth]{images/Chap2/gamma_MCMCanalysis.pdf}
\caption{gamma MCMC analysis}
\label{fig:gamma_MCMCanalysis}
}
\end{figure}

%----------------------------------------------------------------------------------------
%   复杂图表的输入
%----------------------------------------------------------------------------------------
\subsection{复杂图表的输入---tikz}
\label{sec:tikz}
复杂图表最常见的可以算是算法流程图, 可以用standalone独立出一个文档用tikz来画, 生成PDF后作为图片插入
\begin{figure}[ht]
\centering
\includegraphics[width=0.6\textwidth]{images/Chap2/Flowchart.pdf}
\caption{Flowchart}
\label{Flowchart}
\end{figure}

\noindent 此外还有如图\ref{cascade}的类似费曼图那样的示意图, 表示电磁级联的$\gamma-B$过程. 用\LaTeX 的tikz画相对有点复杂但并非不可实现; 有专门的包去画类似这样的费曼图, 但做毕设时我还没学会.
\begin{figure}[ht]
\centering
\includegraphics[width=0.5\textwidth]{images/Chap2/cascade.pdf}
\caption{cascade}
\label{cascade}
\end{figure}

%----------------------------------------------------------------------------------------
%   公式的输入
%----------------------------------------------------------------------------------------
\subsection{公式的输入}
\label{sec:Equations}
一般的公式如下. 如果用户不想要编号 (目的可能是仅仅想摆个不重要的公式, 或者说是推导的中间过程), 可以在equation后面加个*, 即$\backslash$begin\{equation*\}.
\begin{equation}
x=\frac{-b\pm\sqrt{b^2-4ac}}{2a}
\end{equation}
推导的中间过程, 往往需要等号对齐. 可用aligned环境加\&实现 (aligned环境切记每行后面要有换行符$\backslash\backslash$):
\begin{equation}
\begin{aligned}
f(E,\bm{r},t)&=\int\ud^3\bm{r'}\ud t'\ud E'G(\bm{r}, t, E; \bm{r'}, t', E')Q(\bm{r'}, t', E')\\
&=\int\ud^3\bm{r'}\ud t'\ud E'\frac{\delta[t-\tau(E,E')-t']}{b(E)[4\pi\lambda(E, E_0)]^{3/2}}\exp\left[-\frac{(\bm{r}-\bm{r'})^2}{4\lambda(E, E_0)}\right]Q(E')\delta^3(\bm{r'}-\bm{r_0})\delta(t'-t_0)\\
&=\int\ud E'\frac{\delta[t-\tau(E,E')-t_0]}{b(E)[4\pi\lambda(E, E_0)]^{3/2}}\exp\left[-\frac{(\bm{r}-\bm{r_0})^2}{4\lambda(E, E_0)}\right]Q(E')\\
&=\int\ud E'\frac{b(E_0)\delta(E'-E_0)}{b(E)[4\pi\lambda(E, E_0)]^{3/2}}\exp\left[-\frac{(\bm{r}-\bm{r_0})^2}{4\lambda(E, E_0)}\right]Q(E')\\
&=\frac{b(E_0)}{b(E)}\frac{Q(E_0)}{[4\pi\lambda(E, E_0)]^{3/2}}\exp\left[-\frac{(\bm{r}-\bm{r_0})^2}{4\lambda(E, E_0)}\right]\\
&=\frac{E_0^2}{E^2}\frac{Q(E_0)}{[4\pi\lambda(E, E_0)]^{3/2}}\exp\left[-\frac{(\bm{r}-\bm{r_0})^2}{4\lambda(E, E_0)}\right]\\
\end{aligned}
\end{equation}

如果要加个单边大括号, 并实现对齐, 可以适当使用\&, 一个不够就两个, 如Eq \ref{L_MDM}
\begin{equation}\label{L_MDM}
\mathcal{L}=\mathcal{L}_{SM}+\frac{1}{2}\left\{
\begin{aligned}
&\bar{\mathcal{\chi}}({\rm i}\cancel{D}-M)\mathcal{\chi}&&\quad, {\rm for\ fermionic (\mbox{费米子}) }\ \mathcal{\chi}\\
&|D_\mu\mathcal{\chi}|^2-M^2|\mathcal{\chi}|^2&&\quad, {\rm for\ scalar (\mbox{标量}) }\ \mathcal{\chi}\\
\end{aligned}
\right.
\end{equation}
这里, 正体的fermionic和scalar是用了\{$\backslash$rm text\}, ``费米子''和``标量''用了$\backslash$mbox\{中文\}.

关于矩阵, 可以用pmatrix或者bmatrix或者matrix环境. pmatrix自带弯的大括号, bmatrix自带大的中括号[\ \ ], matrix不带括号 (当然你也可以配合$\backslash$left($\backslash$right)加上括号). 一个宏伟的MCMC矩阵方程可以表达为
\begin{small}
\begin{equation}\label{MCMC_equation}
\begin{pmatrix}
p(\theta_{m,t+1,1})\\
\vdots\\
p(\theta_{m,t+1,i})\\
p(\theta_{m,t+1,j})\\
\vdots\\
p(\theta_{m,t+1,S})\\
\end{pmatrix}
=
\begin{pmatrix}
p(\theta_{m,t+1,1}|\theta_{m,t,1})&\cdots&p(\theta_{m,t+1,1}|\theta_{m,t,i})&p(\theta_{m,t+1,1}|\theta_{m,t,j})&\cdots&p(\theta_{m,t+1,1}|\theta_{m,t,S})\\
\vdots&\ddots&\cdots&\cdots&{\mathinner{\mkern2mu\raise1pt\hbox{.}\mkern2mu \raise4pt\hbox{.}\mkern2mu\raise7pt\hbox{.}\mkern1mu}}&\vdots\\
p(\theta_{m,t+1,i}|\theta_{m,t,1})&\cdots&p(\theta_{m,t+1,i}|\theta_{m,t,i})&p(\theta_{m,t+1,i}|\theta_{m,t,j})&\cdots&p(\theta_{m,t+1,i}|\theta_{m,t,S})\\
p(\theta_{m,t+1,j}|\theta_{m,t,1})&\cdots&p(\theta_{m,t+1,j}|\theta_{m,t,i})&p(\theta_{m,t+1,j}|\theta_{m,t,j})&\cdots&p(\theta_{m,t+1,j}|\theta_{m,t,S})\\
\vdots&{\mathinner{\mkern2mu\raise1pt\hbox{.}\mkern2mu \raise4pt\hbox{.}\mkern2mu\raise7pt\hbox{.}\mkern1mu}}&\cdots&\cdots&\ddots&\vdots\\
p(\theta_{m,t+1,S}|\theta_{m,t,1})&\cdots&p(\theta_{m,t+1,S}|\theta_{m,t,i})&p(\theta_{m,t+1,S}|\theta_{m,t,j})&\cdots&p(\theta_{m,t+1,S}|\theta_{m,t,S})\\
\end{pmatrix}
\begin{pmatrix}
p(\theta_{m,t,1})\\
\vdots\\
p(\theta_{m,t,i})\\
p(\theta_{m,t,j})\\
\vdots\\
p(\theta_{m,t,S})\\
\end{pmatrix}
\end{equation}
\end{small}

%----------------------------------------------------------------------------------------
%   表格的插入
%----------------------------------------------------------------------------------------
\subsection{表格的插入}
\label{sec:Tables}
表格通常绘成三线表比较好看. 表格里也可以通过\$a\^2\_1+b\^2\_1=c\^2\_1\$这种语法来插入数学表达式的上下标, 如
\begin{table}[ht]
\centering
\caption{16个脉冲星发射正负电子对的$\eta$}
\begin{tabular}{llllll}
\toprule
Pulsar name&$\tau_c$($\times10^5$ yr)&$|\dot{E}|$($\times10^{33}$ erg/s)&$E_{\rm tot}$($\times10^{47}$ erg)&$E_{\rm out}$($\times10^{47}$ erg)&$\eta$/\%\\
\midrule
J0357+3205&5.40&5.9&54.25&$34^{+186}_{-29}$&$63.62^{+344}_{-53.58}$\\ % Checked
Geminga   &3.42&32&118.03&$13.45^{+0.15}_{-0.15}$&$11.39^{+0.13}_{-0.12}$\\ % Checked
Monogem   &1.11&38&14.76&$3.58^{+0.23}_{-0.21}$&$24.28^{+1.56}_{-1.42}$\\ % Checked
J0736-6304&5.07&0.052&0.42&$21^{+9.6\times10^4}_{-21}$&$5078^{+2.3\times10^7}_{-5077}$\\ % Checked
B0922-52  &3.33&3.4&11.89&$16.1^{+1.0}_{-0.9}$&$135^{+9}_{-8}$\\ % Checked
B0940-55  &4.61&3.1&20.78&$20.53^{+0.89}_{-0.84}$&$98.84^{+4.29}_{-4.03}$\\ % Checked
B0941-56  &3.23&3.0&9.87&$14.82^{+0.72}_{-0.66}$&$150^{+7}_{-7}$\\ % Checked
J0954-5430&1.71&16&14.75&$7.07^{+0.15}_{-0.15}$&$47.93^{+1.03}_{-1.00}$\\ % Checked
B0959-54  &4.43&0.68&4.21&$19.62^{+0.46}_{-0.44}$&$466^{+11}_{-10}$\\ % Checked
B1001-47  &2.20&30&45.79&$9.12^{+0.58}_{-0.54}$&$19.91^{+1.27}_{-1.19}$\\ % Checked
J1020-5921&4.85&0.84&6.23&$36^{+2.6\times10^4}_{-36}$&$584^{+4.2\times10^5}_{-584}$\\ % Checked
B1055-52  &5.35&30&270.79&$22.73^{+0.33}_{-0.32}$&$8.39^{+0.12}_{-0.12}$\\ % Checked
J1732-3131&1.11&150&58.28&$11^{+1.3\times10^4}_{-11}$&$18^{+2.2\times10^3}_{-18}$\\ % Checked
J1741-2054&3.86&9.5&44.64&$16.65^{+0.90}_{-0.85}$&$37.29^{+2.02}_{-1.91}$\\ % Checked
B1742-30  &5.46&8.5&79.91&$24.07^{+0.86}_{-0.83}$&$30.13^{+1.08}_{-1.04}$\\ % Checked
B1822-09  &2.32&4.6&7.80&$8.99^{+0.21}_{-0.19}$&$115.19^{+2.68}_{-2.43}$\\ % Checked
\bottomrule
\end{tabular}
\label{tab:pulsar_eta}
\end{table}

表格当然也可以有行合并和列合并. 列合并的表格如表\ref{tab:Multiple_Pulsar_Complete}.
\begin{table}[t]
\centering
\caption{抛弃简化假设得到的7颗脉冲星的15个关键参数的拟合结果 (含$\chi^2$和置信度)}
\begin{tabular}{lccccc}
\toprule
Pulsar name&$\gamma_s$&$\eta$&$A_{{\rm prim},e^-}$&$\chi^2$&C.I.\\
\midrule
Geminga   &$2.24^{\pm0.19}$&$0.40^{\pm2.6}\%$&\multirow{7}*{$0.69^{\pm0.02}$}&\multirow{7}*{27.24/28}&\multirow{7}*{50.50\%}\\
Monogem   &$1.81^{\pm0.2}$&$8.52^{\pm7.9}\%$&\\
J0954-5430&$2.25^{\pm0.3}$&$1.26^{\pm18}\%$&\\
B1001-47  &$2.41^{\pm0.3}$&$6.37^{\pm6}\%$&\\
B1055-52  &$1.83^{\pm0.4}$&$0.70^{\pm1.3}\%$&\\
J1741-2054&$1.90^{\pm0.36}$&$8.59^{\pm3.2}\%$&\\
B1742-30  &$2.02^{\pm0.13}$&$7.87^{\pm2.3}\%$&\\
\bottomrule
\end{tabular}
\label{tab:Multiple_Pulsar_Complete}
\end{table}

%----------------------------------------------------------------------------------------
%   插入代码块
%----------------------------------------------------------------------------------------
\subsection{插入代码块}
\label{sec:code}
代码块需要用户写好了保存为py文件再插入. \LaTeX 支持大多数主流的编程语言, 若要个性化插入的代码, 需要在本模板的第51-88行自行修改. 代码块的两个例子如Listing 1和2所示
\ \\\ \\
\pythonscript{codes/for}{for循环}\label{for}
\pythonscript{codes/re}{用正则表达式修改文件的特定语句}\label{re}

%----------------------------------------------------------------------------------------
%   字体、列表及脚注
%----------------------------------------------------------------------------------------
\subsection{字体、列表及脚注}
\label{sec:font_list_etc}
\subsubsection{字体}
\label{sec:font_list_etc:font}
中文论文排版中除了宋体, 最常见的用于强调的字体是\textbf{黑体}和\textit{楷体}.
\subsubsection{列表}
\label{sec:font_list_etc:list}
这是一个无序列表
\begin{itemize}
\item 引用文献\cite{long2015fully}, 甚至可以上标引用\upcite{tighe2013finding}. 每引用一篇\LaTeX 就会在末尾自动加多一篇\upcite{hariharan2014simultaneous}.
\item 字体{\color{red}{变红}},\textbf{粗体}
\end{itemize}

这是一个有序列表
\begin{enumerate}
\item 索引前面的章节\ref{sec:Equations}、图\ref{fig:gamma_MCMCanalysis}、表\ref{tab:Multiple_Pulsar_Complete}
\item 加脚注\footnote{https://github.com/OChicken/SCUT-Bachelor-Thesis-Template}
\end{enumerate}

\pagebreak[4]
