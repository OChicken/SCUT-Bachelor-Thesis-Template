%----------------------------------------------------------------------------------------
%   绪论
%----------------------------------------------------------------------------------------
\section{绪论}
%----------------------------------------------------------------------------------------
%   选题背景与意义
%----------------------------------------------------------------------------------------
\subsection{选题背景与意义}
\label{sec:background}
引言是论文正文的开端, 应包括毕业论文选题的背景、目的和意义; 对国内外研究现状和相关领域中已有的研究成果的简要评述; 介绍本项研究工作研究设想、研究方法或实验设计、理论依据或实验基础; 涉及范围和预期结果等. 要求言简意赅,注意不要与摘要雷同或成为摘要的注解.\cite{liu2011sift}

%----------------------------------------------------------------------------------------
%   国内外研究现状和相关工作
%----------------------------------------------------------------------------------------
\subsection{国内外研究现状和相关工作}
\label{sec:related_work}
对国内外研究现状和相关领域中已有的研究成果的简要评述

%----------------------------------------------------------------------------------------
%   本文的论文结构与章节安排
%----------------------------------------------------------------------------------------
\subsection{本文的论文结构与章节安排}
\label{sec:arrangement}
本文共分为五章, 各章节内容安排如下:

第一章引言.

第二章知识点.

第三章方法介绍.

第四章实验和结果.

第五章是本文的最后一章, 总结与展望. 是对本文内容的整体性总结以及对未来工作的展望.

\pagebreak[4]
