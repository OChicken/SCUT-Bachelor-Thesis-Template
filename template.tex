%!mode::"TeX:UTF-8"
% texlive2017, xelatex
% author: Ma Seoyin, 2014 Applied Physics, SCUT
\documentclass[a4paper]{article}
\usepackage[fntef]{ctexcap} % Required for the Chinese and the corresponding section setting
\usepackage[top=2.5cm, bottom=2.5cm, left=2.5cm, right=2.5cm]{geometry} % Required for the Word-like page
\usepackage{fancyhdr} % Required for custom headers
\usepackage{setspace} % Required for the space setting
\usepackage{titlesec} % Required for the Chapter & Section fonts adjustment
\usepackage{titletoc} % Required for the Content fonts adjustment
\usepackage[toc,page]{appendix} % Required for the appendix environment
\usepackage{lastpage} % Required to determine the last page for the footer
\usepackage{extramarks} % Required for headers and footers
\usepackage{courier} % Required for the courier font
\usepackage{float} % Required for the Here float
\usepackage{graphicx} % Required to insert images
\usepackage{wrapfig}
\usepackage{booktabs} % Required for the hline of the three lines table
\usepackage{multirow} % Required for the multirow of table
\usepackage{listings} % Required for insertion of code
\usepackage{indentfirst} % Required for the indent before each paragraph
\usepackage[bookmarks=true,colorlinks=true,linkcolor=black,anchorcolor=blue,citecolor=blue,urlcolor=blue]{hyperref}
%\usepackage{cite} % Required for the ref and cite
\usepackage[usenames,dvipsnames]{color} % Required for custom colors
\usepackage{courier} % Required for the courier font
\usepackage[font=footnotesize,tableposition=top]{caption} % Required for the footnote size captions of figures and tables
\usepackage[numbers,sort&compress]{natbib}
%----------------------------------------------------------------------------------------
%   Math Display
%----------------------------------------------------------------------------------------
\usepackage{bm} % Required for the bold in math display
\usepackage{amsmath} % Required for the math display
\usepackage{amssymb} % Required for the math display
\usepackage{amsbsy} % Required for the math display
\usepackage{cancel} % Required for the cancel symbol in math display
\usepackage{amsthm} % Required for the theorem edition
\usepackage{array} % Required for the array in math display
\usepackage{ifthen} % Required for the conditional commands

%----------------------------------------------------------------------------------------
%   Superscript citation
%----------------------------------------------------------------------------------------
\newcommand{\upcite}[1]{\textsuperscript{\textsuperscript{\cite{#1}}}}

%----------------------------------------------------------------------------------------
%   Upright d in integrate
%----------------------------------------------------------------------------------------
\newcommand{\ud}{\mathrm{d}} 

%----------------------------------------------------------------------------------------
%   Code Inclusion Configuration
%----------------------------------------------------------------------------------------
\definecolor{MyDarkGreen}{rgb}{0.0,0.4,0.0} % This is the color used for comments
\lstloadlanguages{Python} % Load Python syntax for listings, for a list of other languages supported see: ftp://ftp.tex.ac.uk/tex-archive/macros/latex/contrib/listings/listings.pdf
\lstset{language=Python, % Use Python in this example
frame=single, % Single frame around code
%basicstyle=\ttfamily, % Use small true type font
keywordstyle=[1]\color{Blue}\bf, % Python functions bold and blue
keywordstyle=[2]\color{Purple}\it, % Python function arguments purple
keywordstyle=[3]\color{Blue}\underbar, % Custom functions underlined and blue
identifierstyle=, % Nothing special about identifiers                                         
commentstyle=\color{Gray}, % Comments small dark green courier font
stringstyle=\color{Green}, % Strings are purple
showstringspaces=false, % Don't put marks in string spaces
tabsize=4, % 4 spaces per tab
% Put standard Python functions not included in the default language here
morekeywords={rand},
% Put Python function parameters here
morekeywords=[2]{on, off, interp},
% Put user defined functions here
morekeywords=[3]{test},
%
morecomment=[l][\color{Blue}]{...}, % Line continuation (...) like blue comment
numbers=left, % Line numbers on left
firstnumber=1, % Line numbers start with line 1
numberstyle=\tiny\color{Blue}, % Line numbers are blue and small
stepnumber=1 % Line numbers go in steps of 1
}
% Creates a new command to include a Python script, the first parameter is the filename of the script (without .py), the second parameter is the caption
\newcommand{\pythonscript}[2]{
\begin{itemize}
\begin{spacing}{1.0}
\item[]\lstinputlisting[caption=#2,label=#1]{#1.py}
\end{spacing}
\end{itemize}
}

%----------------------------------------------------------------------------------------
%   Section related equations' number
%----------------------------------------------------------------------------------------
\makeatletter
\@addtoreset{equation}{section}
\makeatother
\renewcommand{\theequation}{\arabic{section}.\arabic{equation}}

%----------------------------------------------------------------------------------------
%   Chapter & Section fonts adjustment
%----------------------------------------------------------------------------------------
\CTEXsetup[name={第,章},number={\chinese{section}},format={\centering\zihao{-2}\bfseries}]{section}
\titleformat{\subsection}{\zihao{-3}\bfseries}{\thesubsection}{1em}{}
\titleformat{\subsubsection}{\zihao{4}\bfseries}{\thesubsubsection}{1em}{}
\CTEXsetup[,format={\raggedright\bfseries\zihao{-4}}]{paragraph}
%----------------------------------------------------------------------------------------
%   Content fonts adjustment
%----------------------------------------------------------------------------------------
\renewcommand\contentsname{目\quad\quad 录} % Setup contents
\titlecontents{section}[0em]{\zihao{4}\bfseries}{\contentspush{\thecontentslabel \hspace{0.7em}}}
              {}{\titlerule*[5pt]{.}\contentspage}
\titlecontents{subsection}[2.2em]{\zihao{-4}\songti}{\contentspush{\thecontentslabel\hspace{0.7em}}}
              {}{\titlerule*[5pt]{.}\contentspage}
\titlecontents{subsubsection}[3.9em]{\zihao{-4}\songti}{\contentspush{\thecontentslabel\hspace{0.7em}}}
              {}{\titlerule*[5pt]{.}\contentspage}
\titlespacing*{\subsection} {0pt}{1ex}{1ex} % Adjust the space between title and context
\titlespacing*{\subsubsection} {0pt}{1ex}{1ex}

%----------------------------------------------------------------------------------------
%   Page header & Page footer setting
%----------------------------------------------------------------------------------------
\pagestyle{fancy}
\fancyhf{}
\renewcommand{\sectionmark}[1]{\markboth{第\chinese{section} 章{\color{white}.} #1}{}}
\fancyhead[C]{\CJKfamily{song}华南理工大学学士学位论文}
\fancyhead[CO]{\CJKfamily{song}\leftmark}
\fancyfoot[C]{\thepage}

\begin{document}
%----------------------------------------------------------------------------------------
%   Cover
%----------------------------------------------------------------------------------------
\thispagestyle{empty}
\begin{figure}[ht]
\centering
\includegraphics[height=2.75cm]{title.png}
\end{figure}
\begin{center}
\zihao{0}
\textbf{本科毕业论文}
\end{center}
\nopagebreak[4]
\begin{center}
\zihao{1}
\ \\
\end{center}
\nopagebreak[4]
\begin{center}
\zihao{2}
\textbf{你的本科毕业论文题目}
\end{center}
\nopagebreak[4]
\begin{center}
\zihao{1}
\ \\\ \\\ \\
\end{center}
\nopagebreak[4]
\begin{spacing}{1.8}
\begin{center}
\zihao{-3}
% Via \quad, \qquad or '\ ' (backslash and space) you can adjust any space
\textbf{学\quad\quad 院}\quad\underline{\quad\quad\quad\quad\textbf{你的学院}\quad\quad\quad\quad}\\
\textbf{专\quad\quad 业}\quad\underline{\quad\quad\quad\textbf{你的专业全称}\quad\quad\quad}\\
\textbf{学生姓名}\quad\underline{\quad\quad\quad\quad\textbf{你的名字}\quad\quad\quad\quad}\\
\textbf{学生学号}\quad\underline{\quad\quad\ \ \textbf{201400000000}\ \ \quad\quad}\\
\textbf{指导老师}\quad\underline{\quad\quad\ \textbf{导师1,\ \ \ 导师2}\ \quad\quad}\\
\textbf{提交日期}\quad\underline{\quad\textbf{2000年}\ \ \textbf{00月}\ \ \textbf{00日}\quad}
\end{center}
\end{spacing}
\pagebreak[4]

%----------------------------------------------------------------------------------------
%   Chinese Abstract
%----------------------------------------------------------------------------------------
\setcounter{page}{1}
\pagenumbering{Roman}
\begin{center}
\addcontentsline{toc}{section}{摘要} % Add to content
\zihao{-2}
\bfseries
 摘\quad 要
\end{center}
\thispagestyle{plain}
\begin{spacing}{1.5}
\zihao{-4}
摘要内容应概括地反映出本论文的主要内容, 主要说明本论文的研究目的、内容、方法、成果和结论. 要突出本论文的创造性成果或新见解, 不要与引言相混淆. 语言力求精练、准确,以300-500字为宜. 在摘要的下方另起一行, 注明本文的关键词 (3-5个). 关键词是供检索用的主题词条, 应采用能覆盖论文主要内容的通用技术词条 (参照相应的技术术语标准). 按词条的外延层次排列 (外延大的排在前面). 摘要与关键词应在同一页.
\ \\
\textbf{关键词: }多变量系统; 预测控制; 环境试验设备
\end{spacing}
\pagebreak[4]
\thispagestyle{plain}

%----------------------------------------------------------------------------------------
%   English Abstract
%----------------------------------------------------------------------------------------
\begin{center}
\addcontentsline{toc}{section}{Abstract} % Add to content
\zihao{-2}
Abstract
\end{center}
\begin{spacing}{1.5}
\zihao{-4}
英文摘要内容与中文摘要相同, 以 250-400个实词为宜. 摘要下方另起一行注明英文关键词 (Keywords3-5个).
\ \\
\textbf{Keywords: }Writer recognition; Convolutional Neural Network; Handwritten character recognition
\end{spacing}
\pagebreak[4]

%----------------------------------------------------------------------------------------
%   CONTENTS
%----------------------------------------------------------------------------------------
\thispagestyle{plain}
\begin{spacing}{1.5}
\addcontentsline{toc}{section}{目录}
\tableofcontents
\end{spacing}
\thispagestyle{plain}
\pagebreak[4]

%----------------------------------------------------------------------------------------
%   BEGIN TO COUNT THE PAGE NUMBER
%----------------------------------------------------------------------------------------
\setboolean{@twoside}{true}
\begin{spacing}{1.5}
\zihao{-4}
\setcounter{page}{1}
\pagenumbering{arabic}
%----------------------------------------------------------------------------------------
%   BELOW IS YOUR MAIN TEXT. BEGIN.
%----------------------------------------------------------------------------------------

%----------------------------------------------------------------------------------------
%   绪论
%----------------------------------------------------------------------------------------
\section{绪论}
%----------------------------------------------------------------------------------------
%   选题背景与意义
%----------------------------------------------------------------------------------------
\subsection{选题背景与意义}
\label{sec:background}
引言是论文正文的开端, 应包括毕业论文选题的背景、目的和意义; 对国内外研究现状和相关领域中已有的研究成果的简要评述; 介绍本项研究工作研究设想、研究方法或实验设计、理论依据或实验基础; 涉及范围和预期结果等. 要求言简意赅,注意不要与摘要雷同或成为摘要的注解.

%----------------------------------------------------------------------------------------
%   国内外研究现状和相关工作
%----------------------------------------------------------------------------------------
\subsection{国内外研究现状和相关工作}
\label{sec:related_work}
对国内外研究现状和相关领域中已有的研究成果的简要评述

%----------------------------------------------------------------------------------------
%   本文的论文结构与章节安排
%----------------------------------------------------------------------------------------
\subsection{本文的论文结构与章节安排}
\label{sec:arrangement}
本文共分为五章, 各章节内容安排如下:

第一章引言.

第二章知识点.

第三章方法介绍.

第四章实验和结果.

第五章是本文的最后一章, 总结与展望. 是对本文内容的整体性总结以及对未来工作的展望.

\pagebreak[4]

%----------------------------------------------------------------------------------------
%   简单的使用例子
%----------------------------------------------------------------------------------------
\section{简单的使用例子}
\label{cha:example}
%----------------------------------------------------------------------------------------
%   图片的插入
%----------------------------------------------------------------------------------------
\subsection{图片的插入}
\label{sec:Images}
\begin{wrapfigure}{r}{0.5\linewidth}
\centering
\includegraphics[width=0.5\textwidth]{images/Chap2/confusion.pdf} % width=0.5\textwidth也可以换成width=5cm, height=5cm等关键字参数. width=0.5\textwidth 比较常用: 图片宽度等于你的text宽度的1/2
\caption{镶嵌在文中的图像}
\label{fig:confusion}
\end{wrapfigure}
论文主体是毕业论文的主要部分, 必须言之成理, 论据可靠, 严格遵循本学科国际通行的学术规范. 在写作上要注意结构合理、层次分明、重点突出, 章节标题、公式图表符号必须规范统一. 论文主体的内容根据不同学科有不同的特点, 一般应包括以下几个方面: (1)毕业论文 (设计) 总体方案或选题的论证; (2)毕业论文 (设计) 各部分的设计实现, 包括实验数据的获取、数据可行性及有效性的处理与分析、各部分的设计计算等; (3)对研究内容及成果的客观阐述, 包括理论依据、创新见解、创造性成果及其改进与实际应用价值等; (4)论文主体的所有数据必须真实可靠, 凡引用他人观点、方案、资料、数据等, 无论曾否发表, 无论是纸质或电子版, 均应详加注释. 自然科学论文应推理正确、结论清晰; 人文和社会学科的论文应把握论点正确、论证充分、论据可靠, 恰当运用系统分析和比较研究的方法进行模型或方案设计, 注重实证研究和案例分析, 根据分析结果提出建议和改进措施等.

可以插入单张图像:
\begin{figure}[h]
\centering
\includegraphics[width=0.5\textwidth]{images/Chap2/hole.pdf}
\caption{单张图像}
\label{fig:hole}
\end{figure}

甚至可以两张并排, 如图\ref{B-C_ratio}和图\ref{fig:gamma_MCMCanalysis}所示, 并且这句话已经实现了用$\backslash$ref\{key\}来引用内容. 引用的内容包括公式、表格、图片、代码段等等. 对于参考文献的引用并非用ref, 之后会提及.

对于多张图片的并排并统一标注caption可使用subfigure和makebox; 因为这个功能用得较少, 用户可自行Google.

\begin{figure}[th]
\parbox[t]{0.5\textwidth}{
\includegraphics[width=0.5\textwidth]{images/Chap2/B-C_ratio.pdf}
\caption{B/C ratio}
\label{B-C_ratio}
}
\parbox[t]{0.5\textwidth}{
\includegraphics[width=0.5\textwidth]{images/Chap2/gamma_MCMCanalysis.pdf}
\caption{gamma MCMC analysis}
\label{fig:gamma_MCMCanalysis}
}
\end{figure}

%----------------------------------------------------------------------------------------
%   复杂图表的输入
%----------------------------------------------------------------------------------------
\subsection{复杂图表的输入---tikz}
\label{sec:tikz}
复杂图表最常见的可以算是算法流程图, 可以用standalone独立出一个文档用tikz来画, 生成PDF后作为图片插入
\begin{figure}[ht]
\centering
\includegraphics[width=0.6\textwidth]{images/Chap2/Flowchart.pdf}
\caption{Flowchart}
\label{Flowchart}
\end{figure}

\noindent 此外还有如图\ref{cascade}的类似费曼图那样的示意图, 表示电磁级联的$\gamma-B$过程. 用\LaTeX 的tikz画相对有点复杂但并非不可实现; 有专门的包去画类似这样的费曼图, 但做毕设时我还没学会.
\begin{figure}[ht]
\centering
\includegraphics[width=0.5\textwidth]{images/Chap2/cascade.pdf}
\caption{cascade}
\label{cascade}
\end{figure}

%----------------------------------------------------------------------------------------
%   公式的输入
%----------------------------------------------------------------------------------------
\subsection{公式的输入}
\label{sec:Equations}
一般的公式如下. 如果用户不想要编号 (目的可能是仅仅想摆个不重要的公式, 或者说是推导的中间过程), 可以在equation后面加个*, 即$\backslash$begin\{equation*\}.
\begin{equation}
x=\frac{-b\pm\sqrt{b^2-4ac}}{2a}
\end{equation}
推导的中间过程, 往往需要等号对齐. 可用aligned环境加\&实现 (aligned环境切记每行后面要有换行符$\backslash\backslash$):
\begin{equation}
\begin{aligned}
f(E,\bm{r},t)&=\int\ud^3\bm{r'}\ud t'\ud E'G(\bm{r}, t, E; \bm{r'}, t', E')Q(\bm{r'}, t', E')\\
&=\int\ud^3\bm{r'}\ud t'\ud E'\frac{\delta[t-\tau(E,E')-t']}{b(E)[4\pi\lambda(E, E_0)]^{3/2}}\exp\left[-\frac{(\bm{r}-\bm{r'})^2}{4\lambda(E, E_0)}\right]Q(E')\delta^3(\bm{r'}-\bm{r_0})\delta(t'-t_0)\\
&=\int\ud E'\frac{\delta[t-\tau(E,E')-t_0]}{b(E)[4\pi\lambda(E, E_0)]^{3/2}}\exp\left[-\frac{(\bm{r}-\bm{r_0})^2}{4\lambda(E, E_0)}\right]Q(E')\\
&=\int\ud E'\frac{b(E_0)\delta(E'-E_0)}{b(E)[4\pi\lambda(E, E_0)]^{3/2}}\exp\left[-\frac{(\bm{r}-\bm{r_0})^2}{4\lambda(E, E_0)}\right]Q(E')\\
&=\frac{b(E_0)}{b(E)}\frac{Q(E_0)}{[4\pi\lambda(E, E_0)]^{3/2}}\exp\left[-\frac{(\bm{r}-\bm{r_0})^2}{4\lambda(E, E_0)}\right]\\
&=\frac{E_0^2}{E^2}\frac{Q(E_0)}{[4\pi\lambda(E, E_0)]^{3/2}}\exp\left[-\frac{(\bm{r}-\bm{r_0})^2}{4\lambda(E, E_0)}\right]\\
\end{aligned}
\end{equation}

如果要加个单边大括号, 并实现对齐, 可以适当使用\&, 一个不够就两个, 如Eq \ref{L_MDM}
\begin{equation}\label{L_MDM}
\mathcal{L}=\mathcal{L}_{SM}+\frac{1}{2}\left\{
\begin{aligned}
&\bar{\mathcal{\chi}}({\rm i}\cancel{D}-M)\mathcal{\chi}&&\quad, {\rm for\ fermionic (\mbox{费米子}) }\ \mathcal{\chi}\\
&|D_\mu\mathcal{\chi}|^2-M^2|\mathcal{\chi}|^2&&\quad, {\rm for\ scalar (\mbox{标量}) }\ \mathcal{\chi}\\
\end{aligned}
\right.
\end{equation}
这里, 正体的fermionic和scalar是用了\{$\backslash$rm text\}, ``费米子''和``标量''用了$\backslash$mbox\{中文\}.

关于矩阵, 可以用pmatrix或者bmatrix或者matrix环境. pmatrix自带弯的大括号, bmatrix自带大的中括号[\ \ ], matrix不带括号 (当然你也可以配合$\backslash$left($\backslash$right)加上括号). 一个宏伟的MCMC矩阵方程可以表达为
\begin{small}
\begin{equation}\label{MCMC_equation}
\begin{pmatrix}
p(\theta_{m,t+1,1})\\
\vdots\\
p(\theta_{m,t+1,i})\\
p(\theta_{m,t+1,j})\\
\vdots\\
p(\theta_{m,t+1,S})\\
\end{pmatrix}
=
\begin{pmatrix}
p(\theta_{m,t+1,1}|\theta_{m,t,1})&\cdots&p(\theta_{m,t+1,1}|\theta_{m,t,i})&p(\theta_{m,t+1,1}|\theta_{m,t,j})&\cdots&p(\theta_{m,t+1,1}|\theta_{m,t,S})\\
\vdots&\ddots&\cdots&\cdots&{\mathinner{\mkern2mu\raise1pt\hbox{.}\mkern2mu \raise4pt\hbox{.}\mkern2mu\raise7pt\hbox{.}\mkern1mu}}&\vdots\\
p(\theta_{m,t+1,i}|\theta_{m,t,1})&\cdots&p(\theta_{m,t+1,i}|\theta_{m,t,i})&p(\theta_{m,t+1,i}|\theta_{m,t,j})&\cdots&p(\theta_{m,t+1,i}|\theta_{m,t,S})\\
p(\theta_{m,t+1,j}|\theta_{m,t,1})&\cdots&p(\theta_{m,t+1,j}|\theta_{m,t,i})&p(\theta_{m,t+1,j}|\theta_{m,t,j})&\cdots&p(\theta_{m,t+1,j}|\theta_{m,t,S})\\
\vdots&{\mathinner{\mkern2mu\raise1pt\hbox{.}\mkern2mu \raise4pt\hbox{.}\mkern2mu\raise7pt\hbox{.}\mkern1mu}}&\cdots&\cdots&\ddots&\vdots\\
p(\theta_{m,t+1,S}|\theta_{m,t,1})&\cdots&p(\theta_{m,t+1,S}|\theta_{m,t,i})&p(\theta_{m,t+1,S}|\theta_{m,t,j})&\cdots&p(\theta_{m,t+1,S}|\theta_{m,t,S})\\
\end{pmatrix}
\begin{pmatrix}
p(\theta_{m,t,1})\\
\vdots\\
p(\theta_{m,t,i})\\
p(\theta_{m,t,j})\\
\vdots\\
p(\theta_{m,t,S})\\
\end{pmatrix}
\end{equation}
\end{small}

%----------------------------------------------------------------------------------------
%   表格的插入
%----------------------------------------------------------------------------------------
\subsection{表格的插入}
\label{sec:Tables}
表格通常绘成三线表比较好看. 表格里也可以通过\$a\^2\_1+b\^2\_1=c\^2\_1\$这种语法来插入数学表达式的上下标, 如
\begin{table}[ht]
\centering
\caption{16个脉冲星发射正负电子对的$\eta$}
\begin{tabular}{llllll}
\toprule
Pulsar name&$\tau_c$($\times10^5$ yr)&$|\dot{E}|$($\times10^{33}$ erg/s)&$E_{\rm tot}$($\times10^{47}$ erg)&$E_{\rm out}$($\times10^{47}$ erg)&$\eta$/\%\\
\midrule
J0357+3205&5.40&5.9&54.25&$34^{+186}_{-29}$&$63.62^{+344}_{-53.58}$\\ % Checked
Geminga   &3.42&32&118.03&$13.45^{+0.15}_{-0.15}$&$11.39^{+0.13}_{-0.12}$\\ % Checked
Monogem   &1.11&38&14.76&$3.58^{+0.23}_{-0.21}$&$24.28^{+1.56}_{-1.42}$\\ % Checked
J0736-6304&5.07&0.052&0.42&$21^{+9.6\times10^4}_{-21}$&$5078^{+2.3\times10^7}_{-5077}$\\ % Checked
B0922-52  &3.33&3.4&11.89&$16.1^{+1.0}_{-0.9}$&$135^{+9}_{-8}$\\ % Checked
B0940-55  &4.61&3.1&20.78&$20.53^{+0.89}_{-0.84}$&$98.84^{+4.29}_{-4.03}$\\ % Checked
B0941-56  &3.23&3.0&9.87&$14.82^{+0.72}_{-0.66}$&$150^{+7}_{-7}$\\ % Checked
J0954-5430&1.71&16&14.75&$7.07^{+0.15}_{-0.15}$&$47.93^{+1.03}_{-1.00}$\\ % Checked
B0959-54  &4.43&0.68&4.21&$19.62^{+0.46}_{-0.44}$&$466^{+11}_{-10}$\\ % Checked
B1001-47  &2.20&30&45.79&$9.12^{+0.58}_{-0.54}$&$19.91^{+1.27}_{-1.19}$\\ % Checked
J1020-5921&4.85&0.84&6.23&$36^{+2.6\times10^4}_{-36}$&$584^{+4.2\times10^5}_{-584}$\\ % Checked
B1055-52  &5.35&30&270.79&$22.73^{+0.33}_{-0.32}$&$8.39^{+0.12}_{-0.12}$\\ % Checked
J1732-3131&1.11&150&58.28&$11^{+1.3\times10^4}_{-11}$&$18^{+2.2\times10^3}_{-18}$\\ % Checked
J1741-2054&3.86&9.5&44.64&$16.65^{+0.90}_{-0.85}$&$37.29^{+2.02}_{-1.91}$\\ % Checked
B1742-30  &5.46&8.5&79.91&$24.07^{+0.86}_{-0.83}$&$30.13^{+1.08}_{-1.04}$\\ % Checked
B1822-09  &2.32&4.6&7.80&$8.99^{+0.21}_{-0.19}$&$115.19^{+2.68}_{-2.43}$\\ % Checked
\bottomrule
\end{tabular}
\label{tab:pulsar_eta}
\end{table}

表格当然也可以有行合并和列合并. 列合并的表格如表\ref{tab:Multiple_Pulsar_Complete}.
\begin{table}[t]
\centering
\caption{抛弃简化假设得到的7颗脉冲星的15个关键参数的拟合结果 (含$\chi^2$和置信度)}
\begin{tabular}{lccccc}
\toprule
Pulsar name&$\gamma_s$&$\eta$&$A_{{\rm prim},e^-}$&$\chi^2$&C.I.\\
\midrule
Geminga   &$2.24^{\pm0.19}$&$0.40^{\pm2.6}\%$&\multirow{7}*{$0.69^{\pm0.02}$}&\multirow{7}*{27.24/28}&\multirow{7}*{50.50\%}\\
Monogem   &$1.81^{\pm0.2}$&$8.52^{\pm7.9}\%$&\\
J0954-5430&$2.25^{\pm0.3}$&$1.26^{\pm18}\%$&\\
B1001-47  &$2.41^{\pm0.3}$&$6.37^{\pm6}\%$&\\
B1055-52  &$1.83^{\pm0.4}$&$0.70^{\pm1.3}\%$&\\
J1741-2054&$1.90^{\pm0.36}$&$8.59^{\pm3.2}\%$&\\
B1742-30  &$2.02^{\pm0.13}$&$7.87^{\pm2.3}\%$&\\
\bottomrule
\end{tabular}
\label{tab:Multiple_Pulsar_Complete}
\end{table}

%----------------------------------------------------------------------------------------
%   插入代码块
%----------------------------------------------------------------------------------------
\subsection{插入代码块}
\label{sec:code}
代码块需要用户写好了保存为py文件再插入. \LaTeX 支持大多数主流的编程语言, 若要个性化插入的代码, 需要在本模板的第51-88行自行修改. 代码块的两个例子如Listing 1和2所示
\ \\\ \\
\pythonscript{codes/for}{for循环}\label{for}
\pythonscript{codes/re}{用正则表达式修改文件的特定语句}\label{re}

%----------------------------------------------------------------------------------------
%   字体、列表及脚注
%----------------------------------------------------------------------------------------
\subsection{字体、列表及脚注}
\label{sec:font_list_etc}
\subsubsection{字体}
\label{sec:font_list_etc:font}
中文论文排版中除了宋体, 最常见的用于强调的字体是\textbf{黑体}和\textit{楷体}.
\subsubsection{列表}
\label{sec:font_list_etc:list}
这是一个无序列表
\begin{itemize}
\item 引用文献\cite{long2015fully}, 甚至可以上标引用\upcite{tighe2013finding}. 每引用一篇\LaTeX 就会在末尾自动加多一篇\upcite{hariharan2014simultaneous}.
\item 字体{\color{red}{变红}},\textbf{粗体}
\end{itemize}

这是一个有序列表
\begin{enumerate}
\item 索引前面的章节\ref{sec:Equations}、图\ref{fig:gamma_MCMCanalysis}、表\ref{tab:Multiple_Pulsar_Complete}
\item 加脚注\footnote{https://github.com/OChicken/SCUT-Bachelor-Thesis-Template}
\end{enumerate}

\pagebreak[4]

%----------------------------------------------------------------------------------------
%   页眉
%----------------------------------------------------------------------------------------
\section{页眉}
华工本科毕业论文的一个特色是其页眉也页脚分了奇偶页面来排版. 我将将这句话复制若干次直到下一页, 读者可以看看页眉的变化.

华工本科毕业论文的一个特色是其页眉也页脚分了奇偶页面来排版. 我将将这句话复制若干次直到下一页, 读者可以看看页眉的变化.

华工本科毕业论文的一个特色是其页眉也页脚分了奇偶页面来排版. 我将将这句话复制若干次直到下一页, 读者可以看看页眉的变化.

华工本科毕业论文的一个特色是其页眉也页脚分了奇偶页面来排版. 我将将这句话复制若干次直到下一页, 读者可以看看页眉的变化.

华工本科毕业论文的一个特色是其页眉也页脚分了奇偶页面来排版. 我将将这句话复制若干次直到下一页, 读者可以看看页眉的变化.

华工本科毕业论文的一个特色是其页眉也页脚分了奇偶页面来排版. 我将将这句话复制若干次直到下一页, 读者可以看看页眉的变化.

华工本科毕业论文的一个特色是其页眉也页脚分了奇偶页面来排版. 我将将这句话复制若干次直到下一页, 读者可以看看页眉的变化.

华工本科毕业论文的一个特色是其页眉也页脚分了奇偶页面来排版. 我将将这句话复制若干次直到下一页, 读者可以看看页眉的变化.

华工本科毕业论文的一个特色是其页眉也页脚分了奇偶页面来排版. 我将将这句话复制若干次直到下一页, 读者可以看看页眉的变化.

华工本科毕业论文的一个特色是其页眉也页脚分了奇偶页面来排版. 我将将这句话复制若干次直到下一页, 读者可以看看页眉的变化.

华工本科毕业论文的一个特色是其页眉也页脚分了奇偶页面来排版. 我将将这句话复制若干次直到下一页, 读者可以看看页眉的变化.

华工本科毕业论文的一个特色是其页眉也页脚分了奇偶页面来排版. 我将将这句话复制若干次直到下一页, 读者可以看看页眉的变化.

华工本科毕业论文的一个特色是其页眉也页脚分了奇偶页面来排版. 我将将这句话复制若干次直到下一页, 读者可以看看页眉的变化.

华工本科毕业论文的一个特色是其页眉也页脚分了奇偶页面来排版. 我将将这句话复制若干次直到下一页, 读者可以看看页眉的变化.

华工本科毕业论文的一个特色是其页眉也页脚分了奇偶页面来排版. 我将将这句话复制若干次直到下一页, 读者可以看看页眉的变化.

华工本科毕业论文的一个特色是其页眉也页脚分了奇偶页面来排版. 我将将这句话复制若干次直到下一页, 读者可以看看页眉的变化.

华工本科毕业论文的一个特色是其页眉也页脚分了奇偶页面来排版. 我将将这句话复制若干次直到下一页, 读者可以看看页眉的变化.

华工本科毕业论文的一个特色是其页眉也页脚分了奇偶页面来排版. 我将将这句话复制若干次直到下一页, 读者可以看看页眉的变化.

在接下来的空白章节读者也可以感受一下页眉变化.

\pagebreak[4]

%----------------------------------------------------------------------------------------
%   研究方法
%----------------------------------------------------------------------------------------
\section{研究方法}

\pagebreak[4]

%----------------------------------------------------------------------------------------
%   实验与结果
%----------------------------------------------------------------------------------------
\section{实验与结果}

\pagebreak[4]

%----------------------------------------------------------------------------------------
%   总结与展望
%----------------------------------------------------------------------------------------
\section{总结与展望}

\pagebreak[4]

%----------------------------------------------------------------------------------------
%   APPENDIX
%----------------------------------------------------------------------------------------
\appendix
%----------------------------------------------------------------------------------------
%   Appendix's Section fonts adjustment
%----------------------------------------------------------------------------------------
\CTEXsetup[name={附录},number={\Alph{section}}, format={\centering\zihao{-2}\bfseries}]{section}
\titleformat{\subsection}{\zihao{-3}\bfseries}{\thesubsection}{1em}{}
\titleformat{\subsubsection}{\zihao{4}\bfseries}{\thesubsubsection}{1em}{}
\CTEXsetup[,format={\raggedright\bfseries\zihao{-4}}]{paragraph}
%----------------------------------------------------------------------------------------
%   Page header & Page footer setting
%----------------------------------------------------------------------------------------
\pagestyle{fancy}
\fancyhf{}
\renewcommand{\sectionmark}[1]{\markboth{附录\Alph{section}\ \ #1}{}}
\fancyhead[C]{\CJKfamily{song}华南理工大学学士学位论文}
\fancyhead[CO]{\CJKfamily{song}\leftmark}
\fancyfoot[C]{\thepage}
%----------------------------------------------------------------------------------------
%   Section related equations' number
%----------------------------------------------------------------------------------------
\makeatletter
\@addtoreset{equation}{section}
\makeatother
\renewcommand{\theequation}{\Alph{section}.\arabic{equation}}
%----------------------------------------------------------------------------------------
%   我是附录
%----------------------------------------------------------------------------------------
\section{我是附录}
\label{sec:A}
像``附录A''及其页眉也是可以设置的.

为了展示奇偶页眉的效果, 将上面那句话复制若干次, 直到附录\ref{sec:A}写到第二页位置, 读者可以看看页眉的变化.

像``附录A''及其页眉也是可以设置的.

像``附录A''及其页眉也是可以设置的.

像``附录A''及其页眉也是可以设置的.

像``附录A''及其页眉也是可以设置的.

像``附录A''及其页眉也是可以设置的.

像``附录A''及其页眉也是可以设置的.

像``附录A''及其页眉也是可以设置的.

像``附录A''及其页眉也是可以设置的.

像``附录A''及其页眉也是可以设置的.

像``附录A''及其页眉也是可以设置的.

像``附录A''及其页眉也是可以设置的.

像``附录A''及其页眉也是可以设置的.

像``附录A''及其页眉也是可以设置的.

像``附录A''及其页眉也是可以设置的.

像``附录A''及其页眉也是可以设置的.

像``附录A''及其页眉也是可以设置的.

像``附录A''及其页眉也是可以设置的.

像``附录A''及其页眉也是可以设置的.

像``附录A''及其页眉也是可以设置的.

像``附录A''及其页眉也是可以设置的.

像``附录A''及其页眉也是可以设置的.

像``附录A''及其页眉也是可以设置的.

像``附录A''及其页眉也是可以设置的.

像``附录A''及其页眉也是可以设置的.

像``附录A''及其页眉也是可以设置的.

像``附录A''及其页眉也是可以设置的.

像``附录A''及其页眉也是可以设置的.

像``附录A''及其页眉也是可以设置的.

像``附录A''及其页眉也是可以设置的.

可以试试写个公式. 公式的编号自动为A.1
\begin{equation}\label{equ:A:1}
a^2+b^2=c^2
\end{equation}
并且
\begin{equation}\label{equ:A:2}
c^2=a^2+b^2
\end{equation}

\pagebreak[4]

%----------------------------------------------------------------------------------------
%   我也是附录
%----------------------------------------------------------------------------------------
\section{我也是附录}
\label{sec:B}
到了附录\ref{sec:B}则自动编号为B. 可以试试再写个公式
\begin{equation}
a+b=c
\end{equation}
甚至还能引用A的公式E \ref{equ:A:2}

\pagebreak[4]

\end{spacing}
%----------------------------------------------------------------------------------------
%   ABOVE IS YOUR MAIN TEXT. FINISHED.
%----------------------------------------------------------------------------------------

%----------------------------------------------------------------------------------------
%   参考文献
%----------------------------------------------------------------------------------------
\bibliographystyle{plain}
\addcontentsline{toc}{section}{参考文献}
\zihao{-4}
\bibliography{template}



\pagebreak[4]

%----------------------------------------------------------------------------------------
%   ACKNOWLEDGEMENT
%----------------------------------------------------------------------------------------
\fancyhead[C]{\CJKfamily{song}华南理工大学学士学位论文}
\fancyhead[CO]{\CJKfamily{song}致谢}
\begin{spacing}{1.5}
\section*{致谢}
\addcontentsline{toc}{section}{致谢}
\zihao{-4}
很惭愧, 我只做了一点微小的工作, 谢谢大家.

为了展示奇偶页眉的效果, 将上面那句话复制若干次, 直到致谢写到第二页位置, 读者可以看看页眉的变化.

很惭愧, 就做了一点微小的工作, 谢谢大家.

很惭愧, 就做了一点微小的工作, 谢谢大家.

很惭愧, 就做了一点微小的工作, 谢谢大家.

很惭愧, 就做了一点微小的工作, 谢谢大家.

很惭愧, 就做了一点微小的工作, 谢谢大家.

很惭愧, 就做了一点微小的工作, 谢谢大家.

很惭愧, 就做了一点微小的工作, 谢谢大家.

很惭愧, 就做了一点微小的工作, 谢谢大家.

很惭愧, 就做了一点微小的工作, 谢谢大家.

很惭愧, 就做了一点微小的工作, 谢谢大家.

很惭愧, 就做了一点微小的工作, 谢谢大家.

很惭愧, 就做了一点微小的工作, 谢谢大家.

很惭愧, 就做了一点微小的工作, 谢谢大家.

很惭愧, 就做了一点微小的工作, 谢谢大家.

很惭愧, 就做了一点微小的工作, 谢谢大家.

很惭愧, 就做了一点微小的工作, 谢谢大家.

很惭愧, 就做了一点微小的工作, 谢谢大家.

很惭愧, 就做了一点微小的工作, 谢谢大家.

很惭愧, 就做了一点微小的工作, 谢谢大家.

很惭愧, 就做了一点微小的工作, 谢谢大家.

很惭愧, 就做了一点微小的工作, 谢谢大家.

很惭愧, 就做了一点微小的工作, 谢谢大家.

很惭愧, 就做了一点微小的工作, 谢谢大家.

很惭愧, 就做了一点微小的工作, 谢谢大家.

很惭愧, 就做了一点微小的工作, 谢谢大家.

很惭愧, 就做了一点微小的工作, 谢谢大家.

很惭愧, 就做了一点微小的工作, 谢谢大家.

很惭愧, 就做了一点微小的工作, 谢谢大家.

很惭愧, 就做了一点微小的工作, 谢谢大家.

很惭愧, 就做了一点微小的工作, 谢谢大家.

\end{spacing}
\setboolean{@twoside}{false}

\end{document}